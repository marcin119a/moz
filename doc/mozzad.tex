\documentclass[zad,zawodnik,utf8]{sinol}
\title{Mozaika}
\signature{jrad075}
\id{MOZ}
\etap{zadanie zaliczeniowe 1}
\day{} %to pole zostawiamy puste
\date{20.11.2023, 23:59:59}
\RAM{128}
\iomode{stdin}
\pagestyle{fancy}
\konkurs{Laboratorium z ASD}
\author{Marcin Wierzbiński, Antoni Mikos-Nuszkiewicz}

\begin{document}
  \begin{tasktext}%
    Ostatnio Bajtazar uwielbia tworzyć kolorowe mozaiki z kafelków o pięknych kształtach. Każdy z kafelków może posiadać co najwyżej $k$ detali. Może się zdarzyć że kafelek posiada $0$ detali, co oznacza że nie ma żadnego kształtu.  Jednak Bajtazar ma jedno ważne kryterium - każdy kafelek musi być połączony z sąsiednimi kafelkami i nie może różnić się ilością detali od sąsiednich kafelków o więcej niż jeden detal. Bajtazar uważa, że mozaiki spełniające ten warunek są najpiękniejsze.

    Teraz chcemy dowiedzieć się, ile różnych takich najpiękniejszych mozaik o ustalonych długościach Bajtazar może ułożyć, przy zastrzeżeniu, że interesuje nas tylko reszta z dzielenia przez liczbę $10^{9}$.


    \section{Wejście}
    W pierwszym wierszu wejścia podane są trzy liczby całkowite: $n$, $k$ ($1 \le n, k \le 10^4$), reprezentujące odpowiednio liczbę różnych długości mozaik do sprawdzenia, maksymalna liczba detali na kafelku.

    W kolejnym wierszu znajduje się $n$ liczb całkowitych: $d_0, d_1, \ldots, d_{n-1}$, z zakresu $[1, n]$, które oznaczają kolejne pytania dotyczące liczby najpiękniejszych mozaik.

    \section{Wyjście}
    Twój program powinien wypisać
    na wyjście $n$ liczby całkowitych oddzielonych spacją, oznaczających liczbę najpiękniejszych mozaik modulo $10^{9}$ dla kolejnych ustalonych długości.


    \makecompactexample

    \twocol{natomiast dla danych wejściowych:

    \includefile{../in/\ID1.in}}{poprawnym wynikiem jest:

    \includefile{../out/\ID1.out}}

    \bigskip
    \noindent
    \textbf{Wyjaśnienie do przykładu pierwszego:}
    \begin{table}[h]
      \centering
      \begin{tabular}{|c|c|c|}
        \hline
        \textbf{długość $d_i$} & \textbf{maksymalna liczba detali $k$} & \textbf{najpiękniejsze mozaiki} \\
        \hline
        1 & 1 & $(0), (1)$ \\
        \hline
        2 & 1 & $(0, 0), (0, 1), (1, 0), (1, 1)$ \\
        \hline
        3 & 1 & $(0, 0, 0), (0, 0, 1), (0, 1, 0), (0, 1, 1), (1, 0, 0), (1, 0, 1), (1, 1, 0), (1, 1, 1)$ \\
        \hline
      \end{tabular}
    \end{table}



  \end{tasktext}
\end{document}
