 \documentclass[zad,zawodnik,utf8]{sinol}
  \title{Mozaika}
  \signature{jrad075}
  \id{MOZ}
  \etap{zadanie zaliczeniowe 1}
  \day{} %to pole zostawiamy puste
  \date{20.11.2023, 23:59:59}
  \RAM{128}
  \iomode{stdin}
  \pagestyle{fancy}
  \konkurs{Laboratorium z ASD}
  \author{Marcin Wierzbiński, Antoni Mikos-Nuszkiewicz}

\begin{document}
  \begin{tasktext}%
Ostatnio Bajtazar uwielbia tworzyć kolorowe mozaiki z kafelków o pięknych kształtach, które składają się z $k$ detali. Jednak ma jedno ważne kryterium - każdy kafelek musi być połączony z sąsiednimi kafelkami i nie może różnić się ilością detali od sąsiednich kafelków o więcej niż jeden detal. Bajtazar uważa, że mozaiki spełniające ten warunek są najpiękniejsze.

Teraz chcemy dowiedzieć się, ile różnych takich najpiękniejszych mozaik o ustalonych długościach Bajtazar może ułożyć, przy zastrzeżeniu, że interesuje nas tylko reszta z dzielenia przez pewną liczbę q.


\section{Wejście}
W pierwszym wierszu wejścia podane są trzy liczby całkowite: $n$, $k$ oraz $q$ ($1 \le n, k, q \le 10^3$), reprezentujące odpowiednio liczbę różnych długości mozaik do sprawdzenia, ilość detali oraz wartość $q$ z treści zadania.

W kolejnym wierszu znajduje się $n$ liczb całkowitych: $d_0, d_1, \ldots, d_{n-1}$, z zakresu $[1, n]$, które oznaczają kolejne pytania dotyczące liczby najpiękniejszych mozaik.

\section{Wyjście}
Twój program powinien wypisać
na wyjście $n$ liczby całkowitych oddzielonych spacją, oznaczających liczbę najpiękniejszych mozaik modulo reszta z dzielenia $q$ dla kolejnych ustalonych długości.


\makecompactexample

\bigskip
\noindent
\textbf{Wyjaśnienie do przykładu:}
\begin{table}[h]
\centering
\begin{tabular}{|c|c|c|}
\hline
\textbf{Długość $d_i$} & \textbf{Liczba detali $k$} & \textbf{Najpiękniejsze mozaiki} \\
\hline
1 & 1 & $(0), (1)$ \\
\hline
2 & 1 & $(0, 0), (0, 1), (1, 0), (1, 1)$ \\
\hline
3 & 1 & $(0, 0, 0), (0, 0, 1), (0, 1, 0), (0, 1, 1), (1, 0, 0), (1, 0, 1), (1, 1, 0), (1, 1, 1)$ \\
\hline
\end{tabular}
\end{table}



  \end{tasktext}
\end{document}
